\documentclass[oneside,final,14pt]{extarticle}
\usepackage[utf8]{inputenc}
\usepackage[russianb]{babel}
\usepackage{geometry}
\geometry{
  a4paper,
  left=20mm,top=15mm,right=10mm,bottom=15mm,
  footskip=13mm, includeheadfoot
}
\usepackage{caption}
\captionsetup[figure]{font=small, justification=centering}
\usepackage{graphicx}
\usepackage{mathtools}
\usepackage{indentfirst}
\sloppy
\begin{document}
\thispagestyle{empty}
Разработано программное обеспечение, выполняющее построение эллипса по заданным: точке центра эллипса, двум касательным линиям и соответствующим точкам касания на языке C++ с использованием фреймворка Qt для построения пользовательского интерфейса.

Известно, что для однозначного определения положения эллипса достаточно задать систему из пяти линейно независимых уравнений (например, по пяти точкам, никакие четыре из которых не лежат на одной прямой).

В данной постановке достаточно задать центр эллипса (эквивалентно двум уравнениям), две точки касания и одну касательную. Вторую касательную можно рассматривать как вспомогательную (для проверки, что построение выполнено верно). В системе координат с началом в центре эллипса его всегда можно представить в виде $a_{11}x^2+2a_{12}xy+a_{22}y^2=1$, где $a_{11}>0, a_{22}>0, a_{11}a_{22}-a_{12}^2>0$.

Мы имеем четыре линейных уравнения, даваемых касательными и точками касания, для трёх неизвестных коэффициентов эллипса. Ранг матрицы данной системы может быть равен 3 или 2 (если заданы две противоположные точки эллипса). В первом случае эллипс определяется однозначно при условии, что вторая касательная задана верно. Во втором случае существует бесконечное семейство решений (приложение выводит одно из них).

Следует заметить, что вычисленные значения коэффициентов не всегда будут задавать эллипс. При других знаках инвариантов решение может представлять собой кривую второго порядка другого вида.
\end{document}